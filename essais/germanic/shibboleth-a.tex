% Options for packages loaded elsewhere
\PassOptionsToPackage{unicode}{hyperref}
\PassOptionsToPackage{hyphens}{url}
\documentclass[
  10pt,
  oneside]{book}
\usepackage{xcolor}
\usepackage[margin=2.5cm]{geometry}
\usepackage{amsmath,amssymb}
\setcounter{secnumdepth}{5}
\usepackage{iftex}
\ifPDFTeX
  \usepackage[T1]{fontenc}
  \usepackage[utf8]{inputenc}
  \usepackage{textcomp} % provide euro and other symbols
\else % if luatex or xetex
  \usepackage{unicode-math} % this also loads fontspec
  \defaultfontfeatures{Scale=MatchLowercase}
  \defaultfontfeatures[\rmfamily]{Ligatures=TeX,Scale=1}
\fi
\usepackage{lmodern}
\ifPDFTeX\else
  % xetex/luatex font selection
\fi
% Use upquote if available, for straight quotes in verbatim environments
\IfFileExists{upquote.sty}{\usepackage{upquote}}{}
\IfFileExists{microtype.sty}{% use microtype if available
  \usepackage[]{microtype}
  \UseMicrotypeSet[protrusion]{basicmath} % disable protrusion for tt fonts
}{}
\makeatletter
\@ifundefined{KOMAClassName}{% if non-KOMA class
  \IfFileExists{parskip.sty}{%
    \usepackage{parskip}
  }{% else
    \setlength{\parindent}{0pt}
    \setlength{\parskip}{6pt plus 2pt minus 1pt}}
}{% if KOMA class
  \KOMAoptions{parskip=half}}
\makeatother
\usepackage{longtable,booktabs,array}
\newcounter{none} % for unnumbered tables
\usepackage{calc} % for calculating minipage widths
% Correct order of tables after \paragraph or \subparagraph
\usepackage{etoolbox}
\makeatletter
\patchcmd\longtable{\par}{\if@noskipsec\mbox{}\fi\par}{}{}
\makeatother
% Allow footnotes in longtable head/foot
\IfFileExists{footnotehyper.sty}{\usepackage{footnotehyper}}{\usepackage{footnote}}
\makesavenoteenv{longtable}
\usepackage{graphicx}
\makeatletter
\newsavebox\pandoc@box
\newcommand*\pandocbounded[1]{% scales image to fit in text height/width
  \sbox\pandoc@box{#1}%
  \Gscale@div\@tempa{\textheight}{\dimexpr\ht\pandoc@box+\dp\pandoc@box\relax}%
  \Gscale@div\@tempb{\linewidth}{\wd\pandoc@box}%
  \ifdim\@tempb\p@<\@tempa\p@\let\@tempa\@tempb\fi% select the smaller of both
  \ifdim\@tempa\p@<\p@\scalebox{\@tempa}{\usebox\pandoc@box}%
  \else\usebox{\pandoc@box}%
  \fi%
}
% Set default figure placement to htbp
\def\fps@figure{htbp}
\makeatother
% definitions for citeproc citations
\NewDocumentCommand\citeproctext{}{}
\NewDocumentCommand\citeproc{mm}{%
  \begingroup\def\citeproctext{#2}\cite{#1}\endgroup}
\makeatletter
 % allow citations to break across lines
 \let\@cite@ofmt\@firstofone
 % avoid brackets around text for \cite:
 \def\@biblabel#1{}
 \def\@cite#1#2{{#1\if@tempswa , #2\fi}}
\makeatother
\newlength{\cslhangindent}
\setlength{\cslhangindent}{1.5em}
\newlength{\csllabelwidth}
\setlength{\csllabelwidth}{3em}
\newenvironment{CSLReferences}[2] % #1 hanging-indent, #2 entry-spacing
 {\begin{list}{}{%
  \setlength{\itemindent}{0pt}
  \setlength{\leftmargin}{0pt}
  \setlength{\parsep}{0pt}
  % turn on hanging indent if param 1 is 1
  \ifodd #1
   \setlength{\leftmargin}{\cslhangindent}
   \setlength{\itemindent}{-1\cslhangindent}
  \fi
  % set entry spacing
  \setlength{\itemsep}{#2\baselineskip}}}
 {\end{list}}
\usepackage{calc}
\newcommand{\CSLBlock}[1]{\hfill\break\parbox[t]{\linewidth}{\strut\ignorespaces#1\strut}}
\newcommand{\CSLLeftMargin}[1]{\parbox[t]{\csllabelwidth}{\strut#1\strut}}
\newcommand{\CSLRightInline}[1]{\parbox[t]{\linewidth - \csllabelwidth}{\strut#1\strut}}
\newcommand{\CSLIndent}[1]{\hspace{\cslhangindent}#1}
\setlength{\emergencystretch}{3em} % prevent overfull lines
\providecommand{\tightlist}{%
  \setlength{\itemsep}{0pt}\setlength{\parskip}{0pt}}
\usepackage{fontspec}
\newfontfamily\ipafont{Arial Unicode MS}
\newcommand{\ipa}[1]{{\ipafont #1}}
\usepackage{titlesec}
\titleformat{\chapter}[hang]{\huge\bfseries}{}{0pt}{}
\titleformat{\section}[hang]{\Large\bfseries}{\thesection}{1em}{}
\usepackage{float}
\usepackage{fontspec}
\usepackage{tipa}
\usepackage{bookmark}
\IfFileExists{xurl.sty}{\usepackage{xurl}}{} % add URL line breaks if available
\urlstyle{same}
\hypersetup{
  pdftitle={balkons \& die straszenbahn},
  pdfauthor={st. schwarz},
  hidelinks,
  pdfcreator={LaTeX via pandoc}}

\title{balkons \& die straszenbahn}
\usepackage{etoolbox}
\makeatletter
\providecommand{\subtitle}[1]{% add subtitle to \maketitle
  \apptocmd{\@title}{\par {\large #1 \par}}{}{}
}
\makeatother
\subtitle{cross language observations on The Shibboleth in yiddish, frisian and berlin dialect}
\author{st. schwarz}
\date{2025-11-15}

\begin{document}
\maketitle

\chapter{index}\label{index}

\section{abstract}\label{abstract}

In this paper I want to explore shibboleth phenomena in germanic languages. I'll try to find evidence for shibboleth occurences (which can describe words that are typical in certain registers AND provide the potential to appear differently in phonologic realisation or semantic expression depending on the speaker, making them behave as specific in-group markers) in yiddish and frisian language as well as in berlin vernacular.

\section{inspiration}\label{inspiration}

\subsection{password parsley}\label{password-parsley}

The login to one of the protected pages of a random organisation connected with our university is often carried out using a process known as single sign-on. In this process, the username and password (our ZEDAT \emph{credentials}) are not transmitted to the page we want, but only the confirmation from an instance such that our credentials are correct, i.e., that we have provided the instance (the ``guardian'') the correct username and password to access the remote page (a corpus engine, a library portal, or generally a remote application on the network that only allows access to \emph{affiliates} of certain educational institutions, e.g.) Both sides, guardian and repository, agree that we are only allowed access to the resources if we are who we claim to be. For example, a student at the FUB. Or a partisan. Or a member of a secret society. Or: a confidant of secrets. More precisely: a connoisseur of the SHIBBOLETH.

\subsection{evidence}\label{evidence}

uljana wolf, etymologischer gossip: petersilie: one will no longer be able to use the word unreflected after reading about the 20,000 Haitian guest workers (cf. Wolf (\citeproc{ref-wolf_etymologischer_2021}{2021})). The word petersilie -- parsley -- /\hyperref[tofollow]{perejil}/ is needed as a means of access in order NOT to appear as a stranger to the community of natives = to fall victim to the massacre. Anyone who pronounces it {\ipafont [pɛʟɛχɪʟ]} and not {\ipafont [pɛʀɛχɪʟ]}, as the locals do, will be murdered. So it is good to know the pronunciation. Or to master it.

Something similar, though less drastic, can happen to people who, to name just a few prominent examples, mispronounce \hyperref[tofollow]{derrida}, \hyperref[tofollow]{bourdieu}, \hyperref[tofollow]{accessoir} or, to return to the subject, \hyperref[tofollow]{shibboleth} ({\ipafont  שִׁבֹּלֶת}, cf. Sefaria (\citeproc{ref-sefaria_judges_2025}{2025})) or, to begin the queries, say {\ipafont [balkoːɴ]} instead of {\ipafont [balkɔɳ]} and /tram/ instead of /straszenbahn/.

\section{methods}\label{methods}

Starting from a GPT provided bibliography (disclaimer:of which some entries appear already familiar\ldots{} keywords: shibboleth, group identity, sociolinguistics), I'll dive into corpora provided and intend to consult speakers for their impressions. That may grow to a quantitative corpus based study or either be limited to a qualitative field investigation. I will not only search evidence in natural language but also in literature, which may be easier accessible.

\begin{center}\rule{0.5\linewidth}{0.5pt}\end{center}

\chapter{References}\label{references}

\protect\phantomsection\label{refs}
\begin{CSLReferences}{1}{1}
\bibitem[\citeproctext]{ref-arhammar_beitrage_2000}
Århammar, Nils. 2000. \emph{Beiträge Zur Nordfriesischen {Philologie}}. Nordfriisk Instituut.

\bibitem[\citeproctext]{ref-bailey_dialect_2002}
Bailey, Charles-James. 2002. {``Dialect {Recognition}, {Group} {Boundaries}, and {Shibboleths}.''} \emph{American Speech}.

\bibitem[\citeproctext]{ref-feitsma_saterfrisian_2010}
Feitsma, Anne. 2010. {``Saterfrisian Sociolinguistic Situation.''} \emph{Nordic Journal of Linguistics}.

\bibitem[\citeproctext]{ref-gorter_new_1994}
Gorter, Durk. 1994. {``A New Sociolinguistic Survey of the {Frisian} Language Situation.''} \emph{Journal of Multilingual and Multicultural Development}, ahead of print. \url{https://doi.org/10.1080/03096564.1994.11784030}.

\bibitem[\citeproctext]{ref-heeringa_dialect_2001}
Heeringa, Wilbert, and John Nerbonne. 2001. {``Dialect Distance and Speaker Perception: {The} Case of {Frisian}.''} \emph{Computer {Methods} in {Dialectometry}}.

\bibitem[\citeproctext]{ref-heyen_digitale_2021}
Heyen, H. 2021. \emph{Digitale Nordfriesische {Kommunikation}. \#Hokerbeest?}

\bibitem[\citeproctext]{ref-hoekstra_frisian_2022}
Hoekstra, Jarich. 2022. {``Frisian {Shibboleths}: {Phonological} and {Lexical} {Markers} of {Group} {Identity}.''} \emph{Us Wurk}.

\bibitem[\citeproctext]{ref-katz_shibboleth_2004}
Katz, Dovid. 2004. {``Shibboleth {Phenomena} in {Eastern} {Yiddish}: {Identity} and {Group} {Boundaries}.''} \emph{International Journal of the Sociology of Language}.

\bibitem[\citeproctext]{ref-kvist_prosodic_2019}
Kvist, M. R. 2019. {``Prosodic {Features} of {North} {Frisian} {Dialects}.''} \{PhD\} \{Thesis\}, University of Copenhagen.

\bibitem[\citeproctext]{ref-lonn_shibboleths_2010}
Lönn, Michael. 2010. {``Shibboleths as {Social} {Indexes} in {Germanic} {Languages}.''} \emph{Journal of Germanic Linguistics}.

\bibitem[\citeproctext]{ref-muller_berlin_2018}
Müller, Andrea. 2018. {``Berlin {Dialect} as {Shibboleth}: {Indexical} {Features} in {Urban} {German}.''} \emph{Zeitschrift Für Dialektologie Und Linguistik}.

\bibitem[\citeproctext]{ref-niebuhr_pointed_2015}
Niebuhr, Oliver, and Jarich Hoekstra. 2015. {``Pointed and Plateau-Shaped Pitch Accents in {North} {Frisian} ({Fering}).''} \emph{Linguistische Berichte}, ahead of print. \url{https://doi.org/10.1515/lp-2015-0013}.

\bibitem[\citeproctext]{ref-sefaria_judges_2025}
Sefaria. 2025. \emph{Judges 12.6. {Sefaria}: A {Living} {Library} of {Jewish} {Texts} {Online}}. \url{https://www.sefaria.org/Judges.12.6?lang=bi&with=all&lang2=en}.

\bibitem[\citeproctext]{ref-trudgill_shibboleths_2006}
Trudgill, Peter. 2006. {``Shibboleths and {Social} {Meaning}.''} In \emph{Sociolinguistic {Variation} and {Identity}}. Oxford University Press.

\bibitem[\citeproctext]{ref-voeten_listener_2024}
Voeten, Chris, Anne-Fleur Pinget, Markus Kingma, Nora Stefan, and Hans Van de Velde. 2024. \emph{Listener Factors in Accent Recognition: {A} Perceptual-Dialectology Study of {Frisian}}.

\bibitem[\citeproctext]{ref-wolf_etymologischer_2021}
Wolf, Uljana. 2021. \emph{Etymologischer {Gossip}: {Essays} Und {Reden} / {Uljana} {Wolf}.} 1. Auflage. Kookbooks {Reihe} {Essay} 7. Kookbooks.

\end{CSLReferences}

\end{document}
